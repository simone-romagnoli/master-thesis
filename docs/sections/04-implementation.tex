%% CAPITOLO 4
\clearpage
\chapter{Descrizione delle implementazioni}\label{ch:implementation}
In questo capitolo vengono descritte e discusse le implementazioni delle componenti del sistema.

\section{Manutenzione}\label{sec:mantainance}
Come anticipato nella sezione~\ref{sec:infrastructure}, il progetto è stato mantenuto su \textit{GitHub};
quindi, il \textit{distributed version control system} utilizzato è stato \textbf{Git}.
Per cooperare con i colleghi e garantire allineamento sui branch di sviluppo, è stata seguita la convenzione di \textit{GitFlow} con le seguenti caratteristiche:
\begin{itemize}
    \item Sono stati creati i branch \textit{master} e \textit{develop}, utilizzati rispettivamente per caricare le modifiche sugli ambienti di \textit{staging} (in futuro \textit{produzione}) e \textit{sviluppo}.
    \item Ogni sviluppatore che volesse applicare delle modifiche, deve creare un \textbf{feature branch} a partire da \textit{develop}.
    Il nome del feature branch deve rispettare il pattern: \texttt{feature/<nome-feature>}.
    Eventualmente, si può includere nel nome del feature branch (prima del nome della feature) anche le iniziali dello sviluppatore che se ne occupa.
    Si elencano alcuni esempi di feature branch creati in corso di sviluppo: \texttt{feature/sr-migrations}, \texttt{feature/statistics}, \texttt{feature/openSearchWrite}.
    \item La creazione dei feature branch avviene tramite l'utilizzo del plugin \texttt{gitflow} di \textit{Maven}.
    \item Sono state utilizzate \textit{rebase policy} diverse per quanto riguarda l'allineamento dei feature branch con develop e di develop con master.
    Nello specifico, per allineare i feature branch con develop è stata utilizzata una policy di \textbf{rebase}.
    Invece, per allineare develop con master, è stata utilizzata la rebase policy di \textbf{merge}.
\end{itemize}

\section{Implementazioni}\label{sec:implementations}
In questa sezione vengono descritte le implementazioni realizzate a progetto e a prodotto.
In particolare, ci si concentrerà sulle porzioni di codice ritenute importanti, se non essenziali, per la comprensione dell'architettura del sistema a livello implementativo.
Le porzioni di codice ritenute troppo verbose vengono riportate nell'appendice~\ref{ch:appendix}.

\subsection{Implementazioni del progetto}\label{subsec:project-implementation}
In questa sezione vengono riportati i punti salienti delle implementazioni di ciascun batch del progetto realizzato.

\subsubsection{Data Validation}\label{subsubsec:data}
In questa parte viene descritto il batch che costituisce il punto d'ingresso dei dati per l'intero processo di integrazione.
Inoltre, siccome è il primo capitolo riguardante l'implememtazione delle componenti del sistema, in questa parte verranno fatte diverse descrizioni e assunzioni valide anche per tutti i seguenti batch.

Il listato~\ref{code:main} mostra come sono organizzati i \textit{main} del progetto.
Ognuno rappresenta il punto di ingresso di un batch, per cui è necessario mantenere una determinata struttura.
Infatti, il primo comando consiste nel creare una \textbf{sessione Spark}, necessaria per poter effettuare operazioni distribuite col framework citato.
I comandi successivi rappresentano una serie di chiamate a funzioni di libreria che permettono di caricare in memoria i flussi in base a quanto specificato dalla variabile \texttt{jsonContent}.
Di fatto, quest'ultima carica un file in formato \textit{json} contenente i flussi da leggere in input con i percorsi indicati nel listato~\ref{code:mapping} in appendice.
Questo genere di caricamento permette di specificare lo schema di ciascuna tabella in un formato compatto;
nel dettaglio, per ogni tabella è necessario specificare la chiave primaria ed eventuali chiavi esterne.
In questo modo, nelle fasi successive della validazione sarà possibile controllare se i record sono duplicati o se le chiavi esterne non sono censite.

\begin{lstlisting}[
    label=code:main,
    style=myScalastyle,
    caption="Main del batch di validazione dei flussi in input"
]
object MainData {
    def main(sysArgs: Array[String]): Unit = {
        val spark = SparkSession
          .builder()
          .appName("data-integration")
          .getOrCreate()

        spark.sparkContext.setLogLevel("WARN")

        val config = new Config(spark, sysArgs)

        val jsonContent = scala.io.Source
        .fromResource("conf/ClientMapping.json")
        .mkString
        val batchConfig = Config.readBatchConfig(jsonContent)
        val manager: Manager = config.fromBatchConf(batchConfig, Some(CustomDataTransformations))

        manager.parallelWrite()
    }
}
\end{lstlisting}

Infine, una volta ottenute le configurazioni del caricamento dei flussi, il compito del \texttt{Manager} è quello di caricare effettivamente i dataframe in memoria effettuando delle operazioni di trasformazione, per poi riscrivere il risultato in formato \textit{parquet}.
Le operazioni di trasformazione vengono effettuate in diversi step, seguendo l'ordine dei controlli necessari per la validazione dei record.
In particolare, le trasformazioni vengono definite all'interno dell'oggetto \texttt{CustomDataTransformations} che estende l'interfaccia di libreria \texttt{Transformations}.
Di seguito, si mostra l'interfaccia \texttt{Transformations} per meglio spiegare come avviene l'integrazione dei dataframe.

\clearpage
\begin{lstlisting}[
    label=code:transformations,
    style=myScalastyle,
    caption=Interfaccia con i metodi per realizzare trasformazioni nei diversi step della validazione
]
trait Transformations {
    def beforeDataValidation(config: Config):
        Map[String, mutable.Map[String, OpTable] => Unit]

    def afterDataCast(config: Config):
        Map[String, mutable.Map[String, OpTable] => Unit]

    def afterDuplicatesRemoval(config: Config):
        Map[String, mutable.Map[String, OpTable] => Unit]

    def afterDiscardedRecordsReuse(config: Config):
        Map[String, mutable.Map[String, OpTable] => Unit]

    def afterFkConstraintsCheck(config: Config):
        Map[String, mutable.Map[String, OpTable] => Unit]

    def afterDataValidation(config: Config):
        Map[String, mutable.Map[String, OpTable] => Unit]
}
\end{lstlisting}

Come si può notare dall'interfaccia nel listato~\ref{code:transformations} l'integrazione dei dataframe presenta diversi step;
nello specifico, ogni metodo rappresenta un punto di ingresso per delle trasformazioni dichiarate in maniera funzionale, che vengono eseguite in momenti diversi a seconda del metodo stesso.

\begin{enumerate}
    \item \texttt{beforeDataValidation} - in questo metodo vanno dichiarate tutte le trasformazioni da eseguire prima di qualsiasi step di integrazione.
    Quando si sceglie di effettuare una trasformazione a questo livello, bisogna tener conto del fatto che mancano diverse informazioni, come il tipo di dato di ogni campo (tutti sono caricati come stringhe).
    \item \texttt{afterDataCast} - è la fase immediatamente successiva alla trasformazione dei tipi di dato dei campi.
    Il tipo di dato corretto viene indicato nel mapping, di cui un esempio è indicato nel listato~\ref{code:mapping} in appendice.
    I record che presentano incoerenze con i tipi di dato del mapping vengono scartati (ad esempio, un record con un campo valorizzato con la stringa $"S\grave{\imath}"$ con tipo booleano).
    \item \texttt{afterDuplicatesRemoval} - è la fase immediatamente successiva al controllo delle chiavi primarie.
    Una chiave primaria duplicata significa un record duplicato.
    Nel caso di record duplicati, entrambi i record vanno scartati, in quanto non si sa quale dei due sia quello effettivamente correlato alla chiave primaria.
    \item \texttt{afterDiscardedRecordsReuse} - è la fase immediatamente successiva al riutilizzo di record precedentemente scartati, che sono stati poi corretti.
    \item \texttt{afterFkConstraintsCheck} - è la fase immediatamente successiva al controllo delle chiavi esterne.
    Nello specifico, va controllato che una chiave esterna sia censita, ovvero che esista un record nella tabella $"padre"$ che presenti dei campi valorizzati come la chiave esterna.
    Tale procedura è necessaria in quanto successive operazioni potrebbero provocare dei fallimenti, come, ad esempio, delle operazioni di \textit{join} tra tabelle.
    \item \texttt{afterDataValidation} - in questo metodo vanno dichiarate tutte le trasformazioni da eseguire dopo qualsiasi step di integrazione.
    Le trasformazioni effettuate a questo livello devono tener conto che le informazioni contenute nei dataframe sono complete e che non verranno effettuati altri controlli in seguito.
    Di conseguenza, bisogna operare con cautela per non introdurre errori.
\end{enumerate}
L'ultima parte del batch in questione consiste nella scrittura dei dataframe in un unico formato.
Nello specifico, ogni flusso in output viene riscritto in formato \textit{parquet}.
Tale scelta è dovuta a questioni di \textit{storing} e \textit{performance}.
Infatti, i file \textit{parquet} sono binari compressi che occupano meno spazio su storage rispetto ai file di testo.
Inoltre, i formati di dati basati su colonne permettono query più veloci rispetto ai file di testo~\cite{parquet}.
In questo modo, i batch successivi che dipendono dalla validazione dei flussi possono accedervi in maniera efficiente da un unico punto.

\subsubsection{Scrittura su Database dei Rendimenti}\label{subsubsec:postgres}
Il batch dei rendimenti utilizza in input i flussi validati dal batch di \textit{data validation} e produce dei file \textit{parquet} con informazioni di sintesi relative ai rendimenti ottenuti dai clienti in un certo periodo di tempo.
Quello successivo si occupa di scrivere su un database \textit{Postgres} un sottoinsieme di informazioni relative ai rendimenti calcolati.
Inizialmente, queste due operazioni venivano effettuate in sequenza all'interno di un unico batch.
Una volta effettuato il deploy e lanciato il job su \textit{AWS Glue} in ambiente di sviluppo, ci si è resi conto che la scrittura su database crea un collo di bottiglia rilevante dal punto di vista delle prestazioni, non risolvibile scalando orizzontalmente con il numero di nodi del job.
Nello specifico, il tempo d'esecuzione del processo è più che raddoppiato, passando a oltre 20 minuti rispetto ai 10 precedenti.
Di conseguenza, per permettere di eseguire test in ambiente di sviluppo sui rendimenti calcolati separatamente a quelli scritti su database, è stato deciso di separare la logica in due batch.

Nella realizzazione della scrittura su database \textit{Postgres}, è stata implementata una componente utilizzabile generalmente per ottenere segreti da \textit{AWS Secrets Manager}.
Le informazioni relative alla connessione al database non possono essere inserite nel codice sorgente per motivi di sicurezza.
Quindi vengono mantenute all'interno di un repository di segreti di \textit{AWS Secrets Manager} assieme ad altre informazioni la cui \textit{disclosure} può essere pericolosa.
La componente in questione permette di ottenere un qualsiasi segreto definendone la procedura di deserializzazione.
Nel caso in esempio, il segreto è salvato in formato \textit{json}, per cui è stato deserializzato usando la libreria \textit{jackson} in una classe.

Il listato~\ref{code:secrets-manager} mostra come sia stato reso possibile reperire qualsiasi tipo di segreto.
La funzione \texttt{getSecret} è generica nel tipo di segreto che si vuole ottenere;
questo, assieme alla procedura di deserializzazione (resa implicita), può essere definito al livello del caso d'uso.
In questo modo, è stato possibile inserire la componente per ottenere i segreti tra i package di libreria interna di Prometeia, affinché possa essere riutilizzata.

\clearpage
\begin{lstlisting}[
    label=code:secrets-manager,
    style=myScalastyle,
    caption=Deserializzazione di un segreto da formato json in una apposita classe \texttt{PostgresDbSecret}
]
implicit object PostgresDbSecretProvider extends SecretProvider[PostgresDbSecret] {
  override def secret(secretString: String): PostgresDbSecret =
    new ObjectMapper()
        .registerModule(DefaultScalaModule)
        .readValue(secretString, classOf[PostgresDbSecret])
}

case class PostgresDbSecret(
  @JsonProperty("database") database: String,
  @JsonProperty("driver") driver: String,
  @JsonProperty("endpoint") endpoint: String,
  @JsonProperty("engine") engine: String,
  @JsonProperty("password") password: String,
  @JsonProperty("port") port: String,
  @JsonProperty("schema") schema: String,
  @JsonProperty("username") username: String
) extends SecretsManagerSecret

val dbConnectionSecret = sysArgs("--dbConnectionSecret")
val secret = getSecret[PostgresDbSecret](dbConnectionSecret)
\end{lstlisting}

Un altro aspetto importante per la scrittura dei rendimenti su database è stata l'implementazione di una query per aggiungere informazioni necessarie e rimuovere le superflue.
In particolare, è stata effettuata una operazione di \textit{select} per ridurre il numero di campi da tenere su database e una \textit{filter} per selezionare solo i record di interesse.
Inoltre, è stata definita una \textit{User-Defined Function} (più semplicemente \textit{udf}) che permette di applicare una funzione ai valori di una colonna.
Quest'ultima è stata utilizzata per inizializzare un campo a seconda del valore di un altro, come mostrato nel listato~\ref{code:returns-udf}.

\clearpage
\begin{lstlisting}[
    label=code:returns-udf,
    style=myScalastyle,
    caption=Query per pulire le informazioni da tenere su database \textit{Postgres} con una \textit{User Defined Function}
]
private val performanceBucketUdf = udf(netPerformance => getPerformanceBucket(netPerformance))

private def getPerformanceBucket(netPerformance: Double): Int = netPerformance match {
  case null => null
  case x if x < -0.1 => 1
  case x if x >= -0.1 && x < -0.05 => 2
  case x if x >= -0.05 && x < 0 => 3
  case x if x >= 0 && x < 0.05 => 4
  case x if x >= 0.05 && x < 0.1 => 5
  case _ => 6
}

var returnItems = spark.read.parquet(RETURN_ITEMS_PATH)
returnItems = returnItems
  .filter(col("level") === lit(1))
  .withColumn("performanceBucket", performanceBucketUdf(col("netPerformance")))
  .select("period", "bankCode", "customerCode", "level", "netPerformance", "netReturn", "performanceBucket")
\end{lstlisting}

\subsubsection{Statistiche}\label{subsubsec:stats}
Il batch delle statistiche ha un output leggermente diverso dagli altri, in quanto deve serializzare delle specifiche informazioni che andranno successivamente interpretate da degli \textit{engine}.
Queste informazioni corrispondono a delle richieste convenzionalmente chiamate \textbf{massive}, in quanto vengono invocate per ciascun record del flusso dei clienti.
Lo scopo del batch è quindi quello di prendere in input tutte le informazioni validate in precedenza e calcolare delle specifiche \textbf{statistiche} che andranno poi opportunamente serializzate.

Un aspetto differente rispetto agli altri batch è quello di avere in memoria salvati tutti gli strumenti finanziari.
Solitamente, questi vengono consultati effettuando le opportune trasformazioni sull'apposito \textit{Dataset}, quindi semplicemente costruendo il piano logico di Spark senza mantenere alcuna informazione in memoria locale.
In questo batch, invece, è stato necessario riutilizzare delle porzioni di codice \textit{Java} per la costruzione delle richieste.
Queste porzioni di codice, oltre a non utilizzare la tecnologia Spark, mantengono in memoria le informazioni relative agli strumenti per diverse verifiche, all'interno di un oggetto chiamato \texttt{ProductBO}, ovvero il \textit{business object} degli strumenti.
Per cui, è stato necessario deserializzare gli strumenti dai \textit{DataFrame} in una apposita mappa.
Per farlo, è stata utilizzata l'\textbf{azione} \texttt{collect}:
questa permette di eseguire il piano logico di Spark e collezionare le informazioni in locale per poterle quindi deserializzare.
L'intera procedura è mostrata nel listato~\ref{code:instruments}.

\begin{lstlisting}[
    label=code:instruments,
    style=myScalastyle,
    caption=Invocazione di un'azione per reperire il contenuto di un \textit{Dataset} in locale.
]
// Collect instruments dataframe
val mapping = manager.getColumnMapping("stats")
val mapper = new ObjectMapper()

val instruments = manager.loadedTables("strumenti").df
  .select(mapping:_*)
  .collect()
  .map(row => {
    val instrument = mapper.readValue(row.json, classOf[ProductBO])
    instrument.getCode -> instrument
  }).toMap
\end{lstlisting}

La funzione \texttt{getColumnMapping} preleva da un file di configurazione un insieme di informazioni che permettono di rinominare le colonne di un \textit{Dataset}.
Questo viene fatto per avere un'unica notazione in output.
Convenzionalmente, viene utilizzato il \textit{camel case}, mentre, solitamente, i flussi ricevuti dal cliente arrivano con le colonne scritte in \textit{upper case}.
In questo modo, è possibile definire la trasformazione in un file in formato \textit{json} piuttosto che effettuare una trasformazione per ogni colonna all'interno del codice.
Un esempio di \textit{column mapping} è riportato in appendice nel listato~\ref{code:column-mapping}.

Successivamente, vengono calcolate delle statistiche in senso generico a partire da informazioni sui clienti;
in particolare, per ogni cliente si raggruppano l'insieme dei suoi rapporti con la banca, i saldi relativi a ciascun rapporto e il suo profilo, che riassume i dati sul rischio d'investimento del cliente stesso.
Il listato~\ref{code:stats-joins} mostra come avvengono le trasformazioni di \textit{group by} e \textit{join}.

\begin{lstlisting}[
    label=code:stats-joins,
    style=myScalastyle,
    caption=Raggruppamenti di diversi \textit{Dataset} per quello dei clienti e corrispondenti operazioni di \textit{join}.
]
// Group accounts by customer's primary key
val accountsByCustomer = accounts.df
  .groupBy(customers.primaryKey.map(col):_*)
  .agg(collect_list(struct(accountsMapping:_*)).as("rapporti"))

// Group accosalesnts by customer's primary key
val salesByCustomer = sales.df
  .join(accounts.df, accounts.primaryKey)
  .groupBy(customers.primaryKey.map(col):_*)
  .agg(collect_list(struct(salesMapping:_*)).as("saldi"))

// Apply column mapping to riskProfile
val riskProfile = profile.df
  .withColumn("profiloRischio", struct(profileMapping:_*))

// Join customers with grouped tables
val output = customers.df
  .join(accountsByCustomer, customers.primaryKey, "left")
  .join(riskProfile, customers.primaryKey, "left")
  .join(salesByCustomer, customers.primaryKey, "left")
  .select(
    col("rapporti"),
    col("saldiSottostantiByCliente"),
    col("profiloRischio")
  )
\end{lstlisting}

Si fa notare come, in seguito alla trasformazione di \textit{group by}, sia necessaria un'operazione di aggregazione per riottenere un \textit{Dataset}:
tale trasformazione infatti restituisce un \href{https://spark.apache.org/docs/latest/api/scala/org/apache/spark/sql/RelationalGroupedDataset.html}{\texttt{RelationalGroupedDataset}}.
In questo caso, l'aggregazione corrisponde ad una \texttt{collect\_list}, la quale restituisce una lista di oggetti relativi ad una espressione-colonna.
L'espressione colonna passata come argomento corrisponde al mapping in \textit{camel case} di tutte le colonne dei \textit{Dataset}.
Quindi, in questo modo, si vanno a raggruppare tutti i dati relativi a ciascun cliente, senza tralasciare nessun campo.

La fase successiva di questo batch consiste nella creazione delle richieste per gli \textit{engine} a partire dalle statistiche calcolate.
Per tale scopo viene sfruttato il metodo \texttt{mapPartitions} che permette di applicare una determinata funzione \textbf{ad ogni partizione del \textit{Dataset}}.

\begin{lstlisting}[
    label=code:map-partition,
    style=myScalastyle,
    caption=Invocazione del metodo \texttt{mapPartitions}.
]
val massives: List[MassiveRequest] = List(
  BreakdownAdeguatezza,
  BreakdownCliente,
  BreakdownPosizione
)

val outputSchema = StructType(Seq(
  StructField("type", StringType),
  StructField("value", StringType)
))
implicit val encoder: ExpressionEncoder[Row] = RowEncoder(outputSchema)

stats.mapPartitions(rows => {
  // Create deserialization and serialization object mapper
  val deserializer = new ObjectMapper()
  val xmlSerializer = new XmlMapper()

  // Each row will be mapped into 3 massives
  rows.flatMap(row => {
    val statsInput = deserializer.readValue(row.json, classOf[StatisticheInput])

    // Serialize each massive request into xml code
    massives.map(massive =>
      val request = MassiveRequestBuilder.build(massive, statsInput)
      val xml = xmlSerializer.writeValueAsString(request)
      Row.fromSeq(Seq(massive.toString, xml))
    )
  })
})
\end{lstlisting}

Il metodo \texttt{mapPartitions}, prende come argomento anche un \texttt{Encoder} che specifica la struttura del \textit{Dataset} in output alla funzione di \textit{map}.
Trattandosi di un \textit{DataFrame}, l'\texttt{Encoder} è generico in \texttt{Row} e specifica nomi e tipi delle colonne.
Come si vede nel listato~\ref{code:map-partition}, \texttt{rows} fa riferimento alle righe contenute in ciascuna partizione del \textit{DataFrame} e, per ognuna, vengono poi create le richieste.

\subsubsection{Integrazione Dati \textit{Customer}}\label{subsubsec:customer}
L'ultimo batch della sequenza ha il compito di scrivere su un indice \textit{OpenSearch} le informazioni integrate con le chiamate massive agli \textit{engine}.
Questo perché poi diversi front end e applicativi \textit{as a Service} andranno a prelevare i dati direttamente dall'indice con delle chiamate REST\@.
Per farlo è necessario ottenere prima tutti i dati sui clienti e integrarli con le risposte ottenute dagli \textit{engine} alle chiamate massive.
Una volta fatto ciò, è stato necessario individuare una libreria di AWS aggiornata per l'esecuzione di operazioni \textit{CRUD} sugli indici.
Nello specifico, ogniqualvolta si esegua il batch, questo deve effettuare le seguenti operazioni:
\begin{enumerate}
    \item Creazione di un nuovo indice su cui andranno scritti i nuovi dati integrati e aggiornati.
    \item Rimuovere l'\textit{alias} di \textbf{master} a tutti gli indici precedenti che lo posseggono.
    \item Aggiungere l'\textit{alias} di master al nuovo indice creato.
    \item Eliminare tutti gli indici caricati prima di 10 giorni dalla data corrente.
\end{enumerate}
Gli \textit{alias} vengono gestiti in questo modo in quanto le varie applicazioni che fanno uso degli indici, invocano le chiamate facendo riferimento all'\textit{alias} di \textbf{master}.
In poche parole, si tratta di un meccanismo per avere sempre una versione univoca e aggiornata dei dati sugli indici, riuscendo però a mantenere gli indici dell'ultimo periodo in caso di necessità come eventuali errori o confronti da fare.
In questo modo, l'aggiornamento dell'indice master risulta trasparente ai programmi che invocano chiamate su \textit{OpenSearch} e questi non hanno alcuna dipendenza dalla procedura.
In particolare, gli indici vengono creati con il seguente pattern per quanto riguarda il nome:
\[<tenant>-<environment>-pfp-<time>\]
dove \texttt{tenant} corrisponde al nome del progetto o del cliente a cui si fa riferimento, \texttt{environment} corrisponde all'ambiente su cui sta eseguendo il batch (ad esempio, su ambiente di sviluppo coincide con \textit{dev}) e \texttt{time} corrisponde al \textit{timestamp} del momento in cui viene creato l'indice, in modo da poter capire la cronologia degli indici.
Invece, l'alias di master corrisponde a una stringa creata con il seguente pattern:
\[<tenant>-<environment>-master\]
in modo da avere una gestione \textbf{multi-tenant} degli indici per ogni ambiente di sviluppo.

Per realizzare tutto ciò, è stata implementata una componente che utilizza le librerie di AWS e mette a disposizione dei metodi CRUD per indici e \textit{alias}.
Al momento, tale componente è stata inserita nel progetto, ma per sviluppi futuri potrebbe essere integrata in libreria di Prometeia in modo da poter essere riutilizzata da programmi Scala che devono interagire con indici \textit{OpenSearch}.
Il listato~\ref{code:opensearch} mostra la procedura di creazione di un indice e conseguente scrittura dei record.

\begin{lstlisting}[
    label=code:opensearch,
    style=myScalastyle,
    caption=Creazione di un indice su \textit{OpenSearch} e conseguente scrittura di record.
]
// Create the index
openSearchManager.Indices.create(indexName, classOf[Customer])

// Write all information about the customers
customers.rdd.foreachPartition(rows => {
  // Connect to the index
  val localManager = OpenSearchManager(hostname, port)

  // Deserialize customers into chunks
  val customers = rows.toList.map(row => row.json)
  val bulks = customers.grouped(bulkSize).toList

  // For each chunk, write on index
  bulks.foreach(bulk =>
    localManager.Documents.bulk(indexName, bulk, classOf[Customer])
  )
})
\end{lstlisting}

Si fa notare che, per la scrittura efficiente dei record del \textit{Dataset} contenente i dati integrati sui clienti, viene utilizzata la funzione \texttt{foreachPartition}.
Quest'ultima permette di eseguire la procedura specificata al suo interno in ciascuna partizione del \textit{Dataset}.
In particolare, si crea una connessione con l'indice all'interno di ogni partizione e si deserializzano le informazioni per ciascun record in formato \textit{json}, cioè il formato con cui vengono mantenute le informazioni sugli indici di \textit{OpenSearch}.
In seguito, per garantire una scrittura più efficiente dei record sull'indice, questi vengono divisi in porzioni dette \textit{chunk}.
La creazione dell'indice, oltre al nome dello stesso, richiede di specificare una classe specificata \textit{ad hoc}:
questo perchè gli indici \textit{OpenSearch} necessitano di un \textbf{mapping} che definisca i tipi dei campi al loro interno.
Quindi la classe \texttt{Customer} contiene il modello dei dati (nomi e tipi dei campi) che appariranno sull'indice.
La serializzazione della classe è stata affrontata creando una annotazione apposita, \texttt{OpenSearchMapping}, da utilizzare nei campi di \texttt{Customer} (parzialmente mostrata nel listato~\ref{code:customer}).
Durante la procedura, vengono iterati tramite \textbf{reflection} tutti i campi della classe che hanno tale annotazione:
in questo modo, è stato possibile definire come serializzare ciascun campo con tipo primitivo in un oggetto complesso.
Per i campi con tipo non primitivo è stata applicata la stessa procedura in maniera ricorsiva (ad esempio, per costruire una lista di elementi).

\begin{lstlisting}[
    label=code:customer,
    style=myJavaStyle,
    caption=Classe contente informazioni sui clienti utilizzata per scrivere su indici \textit{OpenSearch} serializzando i campi con l'annotazione \texttt{OpenSearchMapping}.
]
public class Customer extends OpenSearchMapper {

    @OpenSearchMapping(index = false, values = false)
    public String customerCode;

    @OpenSearchMapping(dataType = "text")
    public String fullName;

    @OpenSearchMapping(dataType = "nested", parent = true)
    public List<Balance> balances;
}
\end{lstlisting}

\subsection{Implementazioni di Baseline}\label{subsec:baseline-implementation}
In questa sezione vengono riportati i punti salienti delle implementazioni di ciascun batch realizzato a prodotto.

\subsubsection{Data Validation}\label{subsubsec:baseline-data}
L'implementazione di questo progetto si è limitata alla realizzazione del catalogo degli strumenti.
Quindi inizialmente vengono caricati solo i flussi necessari per ottenere le informazioni del catalogo degli strumenti finanziari di Baseline.
In particolare, sono stati utilizzati flussi \textit{mock}, cioè appositamente creati per sviluppare il programma, ma che comunque rispettano il tracciato di riferimento degli strumenti finanziari:
il primo è stato prodotto internamente, mentre il secondo è stato prelevato dal PFP di un progetto.
Il primo è stato quindi arricchito di informazioni contenute nel secondo e sono state effettuate delle trasformazioni diverse per output differenti.
Nello specifico, per ogni output realizzato è stata effettuata una \textit{left join} tra il flusso prodotto internamente e quello esterno:
in questo modo è stato possibile mantenere le dimensioni del primo, arricchendolo con alcuni dati contenuti solo nell'altro.

Successivamente, viene rieseguito il ciclo di validazione e trasformazione dei flussi a partire da quelli prodotti in output dalla parte precedente (che costituiscono il flusso degli strumenti) e da altri contenenti informazioni da inserire nel catalogo.
A livello architetturale, questa parte corrisponde a quella effettuata per il progetto.
Di conseguenza, si riporta e si commenta nel listato~\ref{code:window} una trasformazione particolare.

\begin{lstlisting}[
    label=code:window,
    style=myScalastyle,
    caption=Applicazione di una \textit{window function}.
]
val window = Window.partitionBy("isin")
  .orderBy(col("value").desc, col("acCode").asc)

val mainRank = mappatura.df
  .withColumn("rank", row_number().over(window))
  .filter(col("rank") === lit(1))
  .select(col("isin"), col("acCode").as("mainCode"))

strumenti.df = strumenti.df
  .join(mainRank, col("code") === col("isin"), "left")
\end{lstlisting}

Il requisito di questa trasformazione corrisponde ad individuare il record con il campo \texttt{value} più elevato per ciascuno strumento (identificato dal codice \textit{isin}) all'interno del flusso \texttt{mappatura}.
Per farlo, è stata realizzata una \textit{window function}, la quale permette di individuare un rango all'interno di un gruppo di record a seconda di un determinato ordinamento.
In questo caso, è stata definita la funzione finestra ordinando i record per il campo desiderato in maniera decrescente;
dopodiché, è stata creata la colonna \texttt{rank} a partire dal numero di riga successivo all'ordinamento.
In questo modo, è stato poi possibile filtrare il \textit{Dataset} per i soli record aventi rango 1, cioè la prima occorrenza di ciascun strumento con il campo \texttt{value} più elevato.

\section{Casi particolari}\label{sec:features}
Questa sezione è dedicata alla descrizione di \textit{feature} particolari, degne di nota, realizzate nei diversi progetti.

\subsection{Migrazioni}\label{subsec:migrations}
La \textit{feature} per le migrazioni è stata complessa da realizzare in quanto porta con sè una serie di complicazioni.
Per migrazioni si intendono cambiamenti di grandi quantità di dati dovuti a eventi specifici o variazioni generiche di informazioni sui clienti finali della banca.
Un esempio molto semplice può essere un cliente che cambia banca:
questa variazione non provoca la perdita di tutti i dati e le informazioni sul cliente, bensì richiede di applicare una modifica;
nel modello trattato, questa informazione è contenuta dalla tabella di anagrafica dei clienti.
Nel caso specificato, si tratterebbe di cambiare il campo \texttt{CODICEBANCA} ed eventualmente il campo \texttt{CODICECLIENTE} nel caso la nuova banca decidesse di assegnarne uno nuovo.
La complessità del caso, in realtà, è dovuta al fatto che un cambiamento di un singolo record in questa tabella richiede l'applicazione della modifica in tutti i flussi che presentano una chiave esterna su di essa:
ad esempio, la tabella sui rapporti di ciascun cliente con la banca, contiene la \textit{foreign key} della tabella dei clienti;
quindi se la chiave del cliente cambia di valore, diventa necessario applicare la variazione anche in quest'altra tabella, in quanto altrimenti si perderebbero le relazioni tra queste.
La situazione diventa maggiormente complessa considerando che alcune migrazioni avvengono su flussi con chiavi molto lunghe, ma soprattutto con molti altri flussi che hanno una dipendenza da essi.

Data la predisposizione dei clienti ad effettuare migrazioni, sia per eventi speciali sia per ordinaria amministrazione, è stato necessario realizzare la \textit{feature} in modo da poterla riutilizzare in occasioni successive.
Grazie a \textit{Scala} è stato possibile implementarla in maniera funzionale, come mostra il listato~\ref{code:migrations}.

\begin{lstlisting}[
    label=code:migrations,
    style=myScalastyle,
    caption=Algoritmo generico di applicazione di una migrazione su un singolo flusso.
]
private def applyMigrations(originalTable: OpTable, migrations: OpTable, joinCondition: Column, ...): DataFrame = {
  // Migrations applied to a stream
  val candidates = originalTable.df.join(migrations.df, joinCondition)

  // Out of all the migrated records, keep only those that are not already existing in the original table
  val toMigrate = candidates
    .join(originalTable.df, originalTable.primaryKey, "leftanti")
    .drop(migrations.df.columns: _*)

  // Filter the original table keeping the records that DO NOT appear in candidates
  val toKeep = originalTable.df.join(
    candidates.drop(migrationFields:_*),
    originalTable.primaryKey,
    "leftanti"
  )

  // Union between migrated and kept
  toKeep.unionByName(toMigrate)
}
\end{lstlisting}

La \textit{feature} è stata inserita in fase di ETL, nello specifico nello step di \textit{afterDataValidation} spiegato nella sezione~\ref{subsubsec:data}:
quindi si considera che tutti i dati trattati in questo punto siano validi, in particolare non duplicati e censiti in tutte le tabelle in relazione tramite chiave esterna.
L'algoritmo individuato prende in input due flussi:
\begin{itemize}
    \item il flusso della tabella su cui va applicata la migrazione (\texttt{originalTable});
    \item il flusso contenente tutte le migrazioni da applicare (\texttt{migrations});
    queste sono contenute all'interno di un flusso dedicato in quanto possono richiedere decine di coppie di campi, dove una coppia contiene il vecchio valore del campo e quello nuovo.
\end{itemize}
Sebbene si possa pensare che può bastare una semplice \textit{join} tra questi, in realtà è necessario prendere delle precauzioni.
Infatti, c'è la possibilità che un record risultato da una migrazione sia già stato erroneamente inserito dalla banca:
in questo caso, l'applicazione della migrazione sul record "vecchio" genererebbe una duplicazione che non verrebbe rimossa visto che tale controllo è già avvenuto.
Quindi, la \textit{join} tra i due flussi serve ad individuare dei candidati per la migrazione, ovvero i record da migrare nel caso ideale in cui non ci siano già dati migrati.
In seguito, la \textit{leftanti join} rimuove dai record migrati quelli già presenti nella tabella originale.
Questo risultato va unito con tutti i dati della tabella originale, esclusi i record in comune con i tutti i candidati precedentemente individuati.
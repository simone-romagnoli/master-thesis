\chapter*{Abstract}
\addcontentsline{toc}{chapter}{Abstract}
\markboth{Abstract}{Abstract}

Questa tesi si incentra sul lavoro svolto durante il tirocinio curriculare presso l'azienda \textit{Prometeia S.P.A}.
\textit{Prometeia} è un provider di servizi di consulenza, soluzioni tecnologiche e ricerca all'avanguardia~\cite{Prometeia-aboutus}.
Il tirocinio si è svolto presso la sede di Bologna.

Il progetto riguarda l'integrazione di dati (anche detta \textit{data integration}) dei clienti finali di una banca.
In particolare, il cliente in questione (cioè la banca) ha richiesto di far convergere tutti i dati dei clienti finali in un unico repository, senza che questi debbano essere prelevati da molteplici sorgenti.
In questo modo, sarebbe possibile avere dei dati omogenei, in un unico formato e in linea con uno standard.
Ad esempio, uno dei casi d'uso di tali dati coincide con la possibilità di poterli mostrare in un'interfaccia grafica come un \textit{front end} web.
Il progetto era già stato realizzato, ma l'evolversi dei dati (in quantità e disomogeneità) ha introdotto la necessità di migliorare le prestazioni del software.
Per sopperire a tale problema si è quindi pensato di adoperare una tecnologia in grado di gestire al meglio i \textit{Big Data}: nello specifico, è stata utilizzata la libreria \textbf{Spark} di \textit{Apache Hadoop}.
La libreria di \textit{Apache Hadoop} consiste in un framework che permette il \textbf{processing distribuito} di grandi dataset su molteplici cluster di computer utilizzando semplici modelli di programmazione~\cite{Apache-hadoop}.
% Tale framework è stato progettato per poter scalare da singoli server a migliaia di macchine, ciascuna dotata di ampie capacità computazionale e di memorizzazione.

Per spiegare al meglio il processo evolutivo del progetto, il lavoro di tesi è stato suddiviso nei seguenti capitoli:

\begin{itemize}
    \item \textbf{Capitolo 1} - Introduzione al problema;
    \item \textbf{Capitolo 2} - Analisi tecnica;
    \item \textbf{Capitolo 3} - Progettazione e descrizione dell'architettura;
    \item \textbf{Capitolo 4} - Descrizione delle implementazioni.
\end{itemize}

\clearpage

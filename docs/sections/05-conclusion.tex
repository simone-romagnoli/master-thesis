%% CONCLUSIONI
\clearpage
\chapter*{Conclusioni}
\addcontentsline{toc}{chapter}{Conclusioni}
\markboth{Conclusioni}{Conclusioni}

In questa tesi è stata riportata l'esperienza di tirocinio in Prometeia, cercando di evidenziare il contributo apportato al progetto.
Il contesto del tirocinio è stato quello di un progetto in partenza, ma con una \textit{timeline} già ben definita;
per cui è stato possibile affrontare i ritmi dell'azienda, imparando già ad adeguarsi alle tempistiche competitive di consegna delle \textit{feature} in progetti importanti.
Vista la dimensione e complessità dei progetti in Prometeia, sarebbe impossibile occuparsi dell'implementazione per l'intero progetto, per cui è stata concordata sin dall'inizio una parte che permettesse un importante sviluppo dell'esperienza:
infatti, è stato possibile studiare una tecnologia relativamente nuova come Spark in ambiente \textit{cloud} con Amazon Web Services.
Quanto appreso ha consentito non solo di affrontare il lavoro da un punto di vista accademico, studiando tecnologie, architetture ed implementazioni in funzione della tesi, ma anche di assumere responsabilità all'interno del team di sviluppo, diventando un punto di riferimento per questioni riguardanti le parti descritte dell'architettura e, in generale, della tecnologia Spark.
Inoltre, il tirocinio è stato un ottimo trampolino di lancio nel mondo del lavoro:
infatti, il rapporto con Prometeia è continuato dopo il termine del tirocinio.

L'utilizzo di tecnologie all'avanguardia come Spark corrisponde ad un salto di qualità per realtà come quelle di Prometeia:
il passaggio al paradigma del \textit{cloud computing} permette un \textit{breakthrough} nelle prestazioni dei servizi forniti, ma anche nell'organizzazione stessa dei team di sviluppo, i quali riescono ad utilizzare risorse in remoto per realizzare software di qualità.
L'analisi delle tecnologie riportata in questa tesi, in particolare di Spark e Amazon Web Services, non ha l'obiettivo di effettuare comparazioni con altri \textit{provider} o altre soluzioni, bensì è stata la conferma di un paradigma che Prometeia ha provato in molteplici contesti.
L'utilizzo di Spark per la realizzazione di Enterprise ETL in combinazione con le capacità, in termini di risorse, dei servizi AWS sono risultati efficienti da molteplici punti di vista.
Il paradigma Spark non è immediato, ma con una buona conoscenza dei sistemi distribuiti e linguaggi moderni, il passaggio dalla formazione all'essere operativi e autonomi è stato veloce.
Inoltre, l'utilizzo di un linguaggio con una sintassi concisa ed elegante come Scala, assieme all'utilizzo di console professionali di AWS, hanno permesso di interpretare facilmente il lavoro svolto da colleghi e di cooperare agilmente con loro, contribuendo in maniera positiva secondo gli standard di Prometeia.

Il tirocinio e il progetto sono iniziati con un approccio accademico, effettuando studi e analisi di problemi e tecnologie, e si sono conclusi con la capacità di lavorare e fare scelte sul campo in completa autonomia.
Si reputa quindi che il tirocinio sia stato il perfetto ponte tra l'università e la carriera lavorativa nel campo dell'ingegneria e delle scienze informatiche.